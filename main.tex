\documentclass[10pt,a4paper,uplatex]{jsarticle}

\usepackage{ascmac}

\begin{document}

%chapter{確率分布と期待値}

\section{期待値}


\begin{itembox}[l]{定義}
  \begin{equation}
    E[g(X)] = \int g(x)f_X(x) dx
  \end{equation}
\end{itembox}

よく混乱するのは、$E[X^2]$ の時に、「$\int x^2f(x^2) dx$ やったっけ?」と思ってしまう(実際は$\int x^2f(x) dx$)。
確率変数が従う確率密度関数 $f_X(x)$ を使用するだけでよい。

\section{確率の関数表示}

離散か連続かで、「確率関数」か「確率密度関数」とう言葉を使い分けている。

\section{母関数}

母関数(generating function)は、何か(確率密度関数、モーメント)を生成することのできる関数である。


\section{確率母関数}

確率密度関数を算出るするためにしようすることができ、一つの確率密度関数と一つの確率母関数(probability generating function)が対応している。

\begin{itembox}[l]{定義}
aaaa
\end{itembox}

\section{積率母関数、特性関数}

積率を算出することの出来る関数

\section{}

\begin{itemize}
  \item 全射:集合$S_1$から$S_2$への射影において、写された$S_1$の像が$S_2$全体である場合。$f(x_1)=x_2$
  \item 単射:集合$S_1$から$S_2$への射影において、$S_1$の任意の元において $x=y$のときにみ $f(x)=f(y)$となる場合。一対一対応。
  \item 全単射:全射かつ単射である写像
\end{itemize}

逆関数とは、$y=f(x)$ を $x=g(y)$ に変形したときの $g(y)$ のことである。
単調増加(減少)である場合に逆関数が定義できる。
全単射の関数の場合に、逆関数が定義できる。


\end{document}
